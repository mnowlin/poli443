\hypertarget{covid-19}{%
\section{COVID-19}\label{covid-19}}

The COVID-19 pandemic is still ongoing. \textbf{The College of
Charleston requires that masks be worn while indoors and you must wear a
mask at all times while in class.} Although vaccinations are currently
not required, \emph{I ask you to be respectful of the health and safety
of others}. If you have not received the \textbf{COVID-19 vaccine, which
is safe, free, and effective, please consider doing so immediately}.
Information about the vaccine is available from the
\href{https://scdhec.gov/covid19/covid-19-vaccine}{SCDHEC website} and
information about where and when to obtain a vaccine is also available
on the SCDHEC website \href{https://vaxlocator.dhec.sc.gov/}{vaccine
locator page}.

\hypertarget{course-description}{%
\section{Course Description}\label{course-description}}

Social-ecological systems are ecological systems that are linked to and
affected by one or more social systems. Management and governance of
such systems include a wide range of stakeholders and institutions. In
this course we will examine various approaches to governing and managing
these systems to make them more resilient.

\vspace{0.10in}

\noindent This course is cross-listed with undergraduate students. As
graduate students you will be expected to lead discussions throughout
the semester.

\vspace{0.1in}

\noindent This course will be \emph{lecture} and \emph{discussion}
based. Being able to adequately participate requires you to come to
class prepared by having done the assigned readings prior to class. In
addition, you should be prepared to participate in class by asking
questions and making informed comments that add to the class discussion.
\textbf{I may call on you to answer a question or discuss your case
studies or research design.}

\vspace{0.1in}

\noindent Laptops are allowed, but discouraged. Phones should be put
away during class. \emph{I encourage you to take notes by hand, with pen
and paper}.
\href{https://www.nytimes.com/2017/11/27/learning/should-teachers-and-professors-ban-student-use-of-laptops-in-class.html}{You
learn better that way}. I recommend taking notes using the
\href{http://www.usu.edu/arc/idea_sheets/pdf/note_taking_cornell.pdf}{Cornell
Method}.

\hypertarget{attendance-policy}{%
\subsection{Attendance Policy}\label{attendance-policy}}

Attendance will be taken for each class session, and will be part of
your course engagement grade. You are allowed to miss \emph{two classes
without penalty}. \textbf{However, do not come to class if you feel
ill.~Additionally, if you have been exposed to or tested positive for
COVID-19, do not come to class regardless of how you feel. In those
cases, I am happy to meet with you on Zoom to discuss material you
missed and wave the attendance requirement. Just let me know.}

\hypertarget{course-goals-and-learning-objectives}{%
\subsection{Course Goals and Learning
Objectives}\label{course-goals-and-learning-objectives}}

After taking this course student will:

\begin{itemize}
\item
  Develop an understanding of social-ecological systems.
\item
  Articulate the challenges involved with the management of common-pool
  resources and social-ecological systems governance.
\item
  Understand the role of institutions in managing the commons and in
  social-ecological systems governance.
\item
  Differentiate the role that institutions and policy play in models of
  social-ecological systems.
\item
  Be able to produce a recommendation to improve governance of a
  social-ecological system.
\end{itemize}

\hypertarget{required-materials}{%
\section{Required Materials}\label{required-materials}}

\begin{itemize}
\item
  \textbf{Books}:

  \begin{itemize}
  
  \item
    Anderies, John M., and Marco A. Janssen. 2016. \emph{Sustaining the
    Commons}. 2nd ed.~Tempe, AZ: Center for Behavior, Institutions, and
    the Environment.

    \begin{itemize}
    
    \item
      A pdf and eBook version of the book is available at the book's
      website,
      \href{https://sustainingthecommons.org/}{sustaingingthecommons.org}
    \end{itemize}
  \item
    Ostrom, Elinor. 2005. \emph{Understanding Institutional Diversity}.
    Princeton, NJ: Princeton University Press.

    \begin{itemize}
    
    \item
      A pdf version of the book is available on
      \href{https://lms.cofc.edu}{OAKS}
    \end{itemize}
  \end{itemize}
\item
  \emph{Other required readings will be provided on}
  \href{https://lms.cofc.edu}{OAKS}
\item
  \href{https://seslibrary.asu.edu/case}{Cases Studies of
  Social-Ecological Systems (Arizona State University)}
\end{itemize}

\hypertarget{course-requirements-and-grading}{%
\section{Course Requirements and
Grading}\label{course-requirements-and-grading}}

Performance in this course will be evaluated on the basis of 8 short
reflection papers, a case study, and course engagement.

\vspace{0.10in}

\noindent Points will be distributed as follows:

\begin{center}
\begin{tabular}{ l l}
\hline
Assignment & Possible Points \\ 
\hline
Reflection papers (8) & 200 points total \\
Case presentations & 100 points total \\
Research design & 400 points total \\
Course engagement & 100 points \\
\hline
Total &  800 points \\
\hline
\end{tabular}
\end{center}

\hypertarget{assignments}{%
\subsection{Assignments}\label{assignments}}

Specific instructions for the following assignments are posted on
\href{https://lms.cofc.edu/d2l/home}{OAKS}. All work must be turned in
through the Assignment folder on OAKS, and is due at class time unless
otherwise specified.

\vspace{0.10in}

\noindent \textbf{Reflection papers}: You will write 8 short, about 2
pages, reflection papers that summarize and integrate the readings.
Prompts will be given for each paper in
\href{https://lms.cofc.edu}{OAKS}. \textbf{When assigned, reflection
papers are due on Mondays at class time.}

\vspace{0.10in}

\noindent \textbf{Case presentations}: Drawing on the library of SES
case studies from Arizona State (or another resource) you will give two
short presentations, about 10 minutes, of two different case studies of
social-ecological systems. Presentation dates will be randomly assigned.

\vspace{0.10in}

\noindent \textbf{Research design}: You will develop a research design
using Ostrom's SES Framework. The proposed research should include
\emph{theoretical} as well as \emph{applied} contributions to scholarly
knowledge. As part of your research design you will develop a research
question, perform a literature review, develop data collection
procedures (e.g., survey/interview questions), state how you would
analyze the data, and conclude with a discussion of the potential
contribution of your proposed research if it was completed. Sections of
the research design will be completed and turned in throughout the
semester. The final research design should be about 20 to 25 pages. You
will give a short presentation, about 15 minutes, of your research
design at the end of the semester.

\vspace{0.10in}

\noindent \textbf{Course engagement}: Course engagement involves coming
to class prepared, having read, and ready with questions and comments.

\hypertarget{late-work-policy}{%
\subsection{Late Work Policy}\label{late-work-policy}}

Late work is subject to a 48-hour grace period, and after that will be
penalized 10\% each day (24 hr period) it is late, up to 3 days. After 3
days the assignment will not be accepted. For example, if an assignment
is due Wednesday at 2:00 PM, the grace period ends on Friday at 2:00 PM
and it is late as of 2:01 PM and you lose 10\%. After Saturday at 2:01
PM you lose another 10\%, after Sunday at 2:01 PM another 10\%, and no
work will be accepted after Monday at 2:00 PM. \emph{No late work will
accepted 72 hrs after the assignment due date and time}.

\hypertarget{grading-scale}{%
\subsection{Grading Scale}\label{grading-scale}}

There are \textbf{800} possible points for this course. Grades will be
allocated based on your earned points and calculated as a percentage of
\textbf{800}. A: 94 to 100\%; A-: 90 to 93\%; B+: 87 to 89\%; B: 83 to
86\%; B-: 80 to 82\%; C+: 77 to 79\%; C: 73 to 76\%; C-: 70 to 72\%; D+:
67 to 69\%; D: 63 to 67\%; D-: 60 to 62\%; F: 59\% and below
